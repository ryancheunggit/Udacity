
% Default to the notebook output style

    


% Inherit from the specified cell style.




    
\documentclass{article}

    
    
    \usepackage{graphicx} % Used to insert images
    \usepackage{adjustbox} % Used to constrain images to a maximum size 
    \usepackage{color} % Allow colors to be defined
    \usepackage{enumerate} % Needed for markdown enumerations to work
    \usepackage{geometry} % Used to adjust the document margins
    \usepackage{amsmath} % Equations
    \usepackage{amssymb} % Equations
    \usepackage[mathletters]{ucs} % Extended unicode (utf-8) support
    \usepackage[utf8x]{inputenc} % Allow utf-8 characters in the tex document
    \usepackage{fancyvrb} % verbatim replacement that allows latex
    \usepackage{grffile} % extends the file name processing of package graphics 
                         % to support a larger range 
    % The hyperref package gives us a pdf with properly built
    % internal navigation ('pdf bookmarks' for the table of contents,
    % internal cross-reference links, web links for URLs, etc.)
    \usepackage{hyperref}
    \usepackage{longtable} % longtable support required by pandoc >1.10
    \usepackage{booktabs}  % table support for pandoc > 1.12.2
    

    
    
    \definecolor{orange}{cmyk}{0,0.4,0.8,0.2}
    \definecolor{darkorange}{rgb}{.71,0.21,0.01}
    \definecolor{darkgreen}{rgb}{.12,.54,.11}
    \definecolor{myteal}{rgb}{.26, .44, .56}
    \definecolor{gray}{gray}{0.45}
    \definecolor{lightgray}{gray}{.95}
    \definecolor{mediumgray}{gray}{.8}
    \definecolor{inputbackground}{rgb}{.95, .95, .85}
    \definecolor{outputbackground}{rgb}{.95, .95, .95}
    \definecolor{traceback}{rgb}{1, .95, .95}
    % ansi colors
    \definecolor{red}{rgb}{.6,0,0}
    \definecolor{green}{rgb}{0,.65,0}
    \definecolor{brown}{rgb}{0.6,0.6,0}
    \definecolor{blue}{rgb}{0,.145,.698}
    \definecolor{purple}{rgb}{.698,.145,.698}
    \definecolor{cyan}{rgb}{0,.698,.698}
    \definecolor{lightgray}{gray}{0.5}
    
    % bright ansi colors
    \definecolor{darkgray}{gray}{0.25}
    \definecolor{lightred}{rgb}{1.0,0.39,0.28}
    \definecolor{lightgreen}{rgb}{0.48,0.99,0.0}
    \definecolor{lightblue}{rgb}{0.53,0.81,0.92}
    \definecolor{lightpurple}{rgb}{0.87,0.63,0.87}
    \definecolor{lightcyan}{rgb}{0.5,1.0,0.83}
    
    % commands and environments needed by pandoc snippets
    % extracted from the output of `pandoc -s`
    \DefineVerbatimEnvironment{Highlighting}{Verbatim}{commandchars=\\\{\}}
    % Add ',fontsize=\small' for more characters per line
    \newenvironment{Shaded}{}{}
    \newcommand{\KeywordTok}[1]{\textcolor[rgb]{0.00,0.44,0.13}{\textbf{{#1}}}}
    \newcommand{\DataTypeTok}[1]{\textcolor[rgb]{0.56,0.13,0.00}{{#1}}}
    \newcommand{\DecValTok}[1]{\textcolor[rgb]{0.25,0.63,0.44}{{#1}}}
    \newcommand{\BaseNTok}[1]{\textcolor[rgb]{0.25,0.63,0.44}{{#1}}}
    \newcommand{\FloatTok}[1]{\textcolor[rgb]{0.25,0.63,0.44}{{#1}}}
    \newcommand{\CharTok}[1]{\textcolor[rgb]{0.25,0.44,0.63}{{#1}}}
    \newcommand{\StringTok}[1]{\textcolor[rgb]{0.25,0.44,0.63}{{#1}}}
    \newcommand{\CommentTok}[1]{\textcolor[rgb]{0.38,0.63,0.69}{\textit{{#1}}}}
    \newcommand{\OtherTok}[1]{\textcolor[rgb]{0.00,0.44,0.13}{{#1}}}
    \newcommand{\AlertTok}[1]{\textcolor[rgb]{1.00,0.00,0.00}{\textbf{{#1}}}}
    \newcommand{\FunctionTok}[1]{\textcolor[rgb]{0.02,0.16,0.49}{{#1}}}
    \newcommand{\RegionMarkerTok}[1]{{#1}}
    \newcommand{\ErrorTok}[1]{\textcolor[rgb]{1.00,0.00,0.00}{\textbf{{#1}}}}
    \newcommand{\NormalTok}[1]{{#1}}
    
    % Define a nice break command that doesn't care if a line doesn't already
    % exist.
    \def\br{\hspace*{\fill} \\* }
    % Math Jax compatability definitions
    \def\gt{>}
    \def\lt{<}
    % Document parameters
    \title{Analyzing the NYC Subway Dataset}
    
    
    

    % Pygments definitions
    
\makeatletter
\def\PY@reset{\let\PY@it=\relax \let\PY@bf=\relax%
    \let\PY@ul=\relax \let\PY@tc=\relax%
    \let\PY@bc=\relax \let\PY@ff=\relax}
\def\PY@tok#1{\csname PY@tok@#1\endcsname}
\def\PY@toks#1+{\ifx\relax#1\empty\else%
    \PY@tok{#1}\expandafter\PY@toks\fi}
\def\PY@do#1{\PY@bc{\PY@tc{\PY@ul{%
    \PY@it{\PY@bf{\PY@ff{#1}}}}}}}
\def\PY#1#2{\PY@reset\PY@toks#1+\relax+\PY@do{#2}}

\expandafter\def\csname PY@tok@gd\endcsname{\def\PY@tc##1{\textcolor[rgb]{0.63,0.00,0.00}{##1}}}
\expandafter\def\csname PY@tok@gu\endcsname{\let\PY@bf=\textbf\def\PY@tc##1{\textcolor[rgb]{0.50,0.00,0.50}{##1}}}
\expandafter\def\csname PY@tok@gt\endcsname{\def\PY@tc##1{\textcolor[rgb]{0.00,0.27,0.87}{##1}}}
\expandafter\def\csname PY@tok@gs\endcsname{\let\PY@bf=\textbf}
\expandafter\def\csname PY@tok@gr\endcsname{\def\PY@tc##1{\textcolor[rgb]{1.00,0.00,0.00}{##1}}}
\expandafter\def\csname PY@tok@cm\endcsname{\let\PY@it=\textit\def\PY@tc##1{\textcolor[rgb]{0.25,0.50,0.50}{##1}}}
\expandafter\def\csname PY@tok@vg\endcsname{\def\PY@tc##1{\textcolor[rgb]{0.10,0.09,0.49}{##1}}}
\expandafter\def\csname PY@tok@m\endcsname{\def\PY@tc##1{\textcolor[rgb]{0.40,0.40,0.40}{##1}}}
\expandafter\def\csname PY@tok@mh\endcsname{\def\PY@tc##1{\textcolor[rgb]{0.40,0.40,0.40}{##1}}}
\expandafter\def\csname PY@tok@go\endcsname{\def\PY@tc##1{\textcolor[rgb]{0.53,0.53,0.53}{##1}}}
\expandafter\def\csname PY@tok@ge\endcsname{\let\PY@it=\textit}
\expandafter\def\csname PY@tok@vc\endcsname{\def\PY@tc##1{\textcolor[rgb]{0.10,0.09,0.49}{##1}}}
\expandafter\def\csname PY@tok@il\endcsname{\def\PY@tc##1{\textcolor[rgb]{0.40,0.40,0.40}{##1}}}
\expandafter\def\csname PY@tok@cs\endcsname{\let\PY@it=\textit\def\PY@tc##1{\textcolor[rgb]{0.25,0.50,0.50}{##1}}}
\expandafter\def\csname PY@tok@cp\endcsname{\def\PY@tc##1{\textcolor[rgb]{0.74,0.48,0.00}{##1}}}
\expandafter\def\csname PY@tok@gi\endcsname{\def\PY@tc##1{\textcolor[rgb]{0.00,0.63,0.00}{##1}}}
\expandafter\def\csname PY@tok@gh\endcsname{\let\PY@bf=\textbf\def\PY@tc##1{\textcolor[rgb]{0.00,0.00,0.50}{##1}}}
\expandafter\def\csname PY@tok@ni\endcsname{\let\PY@bf=\textbf\def\PY@tc##1{\textcolor[rgb]{0.60,0.60,0.60}{##1}}}
\expandafter\def\csname PY@tok@nl\endcsname{\def\PY@tc##1{\textcolor[rgb]{0.63,0.63,0.00}{##1}}}
\expandafter\def\csname PY@tok@nn\endcsname{\let\PY@bf=\textbf\def\PY@tc##1{\textcolor[rgb]{0.00,0.00,1.00}{##1}}}
\expandafter\def\csname PY@tok@no\endcsname{\def\PY@tc##1{\textcolor[rgb]{0.53,0.00,0.00}{##1}}}
\expandafter\def\csname PY@tok@na\endcsname{\def\PY@tc##1{\textcolor[rgb]{0.49,0.56,0.16}{##1}}}
\expandafter\def\csname PY@tok@nb\endcsname{\def\PY@tc##1{\textcolor[rgb]{0.00,0.50,0.00}{##1}}}
\expandafter\def\csname PY@tok@nc\endcsname{\let\PY@bf=\textbf\def\PY@tc##1{\textcolor[rgb]{0.00,0.00,1.00}{##1}}}
\expandafter\def\csname PY@tok@nd\endcsname{\def\PY@tc##1{\textcolor[rgb]{0.67,0.13,1.00}{##1}}}
\expandafter\def\csname PY@tok@ne\endcsname{\let\PY@bf=\textbf\def\PY@tc##1{\textcolor[rgb]{0.82,0.25,0.23}{##1}}}
\expandafter\def\csname PY@tok@nf\endcsname{\def\PY@tc##1{\textcolor[rgb]{0.00,0.00,1.00}{##1}}}
\expandafter\def\csname PY@tok@si\endcsname{\let\PY@bf=\textbf\def\PY@tc##1{\textcolor[rgb]{0.73,0.40,0.53}{##1}}}
\expandafter\def\csname PY@tok@s2\endcsname{\def\PY@tc##1{\textcolor[rgb]{0.73,0.13,0.13}{##1}}}
\expandafter\def\csname PY@tok@vi\endcsname{\def\PY@tc##1{\textcolor[rgb]{0.10,0.09,0.49}{##1}}}
\expandafter\def\csname PY@tok@nt\endcsname{\let\PY@bf=\textbf\def\PY@tc##1{\textcolor[rgb]{0.00,0.50,0.00}{##1}}}
\expandafter\def\csname PY@tok@nv\endcsname{\def\PY@tc##1{\textcolor[rgb]{0.10,0.09,0.49}{##1}}}
\expandafter\def\csname PY@tok@s1\endcsname{\def\PY@tc##1{\textcolor[rgb]{0.73,0.13,0.13}{##1}}}
\expandafter\def\csname PY@tok@kd\endcsname{\let\PY@bf=\textbf\def\PY@tc##1{\textcolor[rgb]{0.00,0.50,0.00}{##1}}}
\expandafter\def\csname PY@tok@sh\endcsname{\def\PY@tc##1{\textcolor[rgb]{0.73,0.13,0.13}{##1}}}
\expandafter\def\csname PY@tok@sc\endcsname{\def\PY@tc##1{\textcolor[rgb]{0.73,0.13,0.13}{##1}}}
\expandafter\def\csname PY@tok@sx\endcsname{\def\PY@tc##1{\textcolor[rgb]{0.00,0.50,0.00}{##1}}}
\expandafter\def\csname PY@tok@bp\endcsname{\def\PY@tc##1{\textcolor[rgb]{0.00,0.50,0.00}{##1}}}
\expandafter\def\csname PY@tok@c1\endcsname{\let\PY@it=\textit\def\PY@tc##1{\textcolor[rgb]{0.25,0.50,0.50}{##1}}}
\expandafter\def\csname PY@tok@kc\endcsname{\let\PY@bf=\textbf\def\PY@tc##1{\textcolor[rgb]{0.00,0.50,0.00}{##1}}}
\expandafter\def\csname PY@tok@c\endcsname{\let\PY@it=\textit\def\PY@tc##1{\textcolor[rgb]{0.25,0.50,0.50}{##1}}}
\expandafter\def\csname PY@tok@mf\endcsname{\def\PY@tc##1{\textcolor[rgb]{0.40,0.40,0.40}{##1}}}
\expandafter\def\csname PY@tok@err\endcsname{\def\PY@bc##1{\setlength{\fboxsep}{0pt}\fcolorbox[rgb]{1.00,0.00,0.00}{1,1,1}{\strut ##1}}}
\expandafter\def\csname PY@tok@mb\endcsname{\def\PY@tc##1{\textcolor[rgb]{0.40,0.40,0.40}{##1}}}
\expandafter\def\csname PY@tok@ss\endcsname{\def\PY@tc##1{\textcolor[rgb]{0.10,0.09,0.49}{##1}}}
\expandafter\def\csname PY@tok@sr\endcsname{\def\PY@tc##1{\textcolor[rgb]{0.73,0.40,0.53}{##1}}}
\expandafter\def\csname PY@tok@mo\endcsname{\def\PY@tc##1{\textcolor[rgb]{0.40,0.40,0.40}{##1}}}
\expandafter\def\csname PY@tok@kn\endcsname{\let\PY@bf=\textbf\def\PY@tc##1{\textcolor[rgb]{0.00,0.50,0.00}{##1}}}
\expandafter\def\csname PY@tok@mi\endcsname{\def\PY@tc##1{\textcolor[rgb]{0.40,0.40,0.40}{##1}}}
\expandafter\def\csname PY@tok@gp\endcsname{\let\PY@bf=\textbf\def\PY@tc##1{\textcolor[rgb]{0.00,0.00,0.50}{##1}}}
\expandafter\def\csname PY@tok@o\endcsname{\def\PY@tc##1{\textcolor[rgb]{0.40,0.40,0.40}{##1}}}
\expandafter\def\csname PY@tok@kr\endcsname{\let\PY@bf=\textbf\def\PY@tc##1{\textcolor[rgb]{0.00,0.50,0.00}{##1}}}
\expandafter\def\csname PY@tok@s\endcsname{\def\PY@tc##1{\textcolor[rgb]{0.73,0.13,0.13}{##1}}}
\expandafter\def\csname PY@tok@kp\endcsname{\def\PY@tc##1{\textcolor[rgb]{0.00,0.50,0.00}{##1}}}
\expandafter\def\csname PY@tok@w\endcsname{\def\PY@tc##1{\textcolor[rgb]{0.73,0.73,0.73}{##1}}}
\expandafter\def\csname PY@tok@kt\endcsname{\def\PY@tc##1{\textcolor[rgb]{0.69,0.00,0.25}{##1}}}
\expandafter\def\csname PY@tok@ow\endcsname{\let\PY@bf=\textbf\def\PY@tc##1{\textcolor[rgb]{0.67,0.13,1.00}{##1}}}
\expandafter\def\csname PY@tok@sb\endcsname{\def\PY@tc##1{\textcolor[rgb]{0.73,0.13,0.13}{##1}}}
\expandafter\def\csname PY@tok@k\endcsname{\let\PY@bf=\textbf\def\PY@tc##1{\textcolor[rgb]{0.00,0.50,0.00}{##1}}}
\expandafter\def\csname PY@tok@se\endcsname{\let\PY@bf=\textbf\def\PY@tc##1{\textcolor[rgb]{0.73,0.40,0.13}{##1}}}
\expandafter\def\csname PY@tok@sd\endcsname{\let\PY@it=\textit\def\PY@tc##1{\textcolor[rgb]{0.73,0.13,0.13}{##1}}}

\def\PYZbs{\char`\\}
\def\PYZus{\char`\_}
\def\PYZob{\char`\{}
\def\PYZcb{\char`\}}
\def\PYZca{\char`\^}
\def\PYZam{\char`\&}
\def\PYZlt{\char`\<}
\def\PYZgt{\char`\>}
\def\PYZsh{\char`\#}
\def\PYZpc{\char`\%}
\def\PYZdl{\char`\$}
\def\PYZhy{\char`\-}
\def\PYZsq{\char`\'}
\def\PYZdq{\char`\"}
\def\PYZti{\char`\~}
% for compatibility with earlier versions
\def\PYZat{@}
\def\PYZlb{[}
\def\PYZrb{]}
\makeatother


    % Exact colors from NB
    \definecolor{incolor}{rgb}{0.0, 0.0, 0.5}
    \definecolor{outcolor}{rgb}{0.545, 0.0, 0.0}



    
    % Prevent overflowing lines due to hard-to-break entities
    \sloppy 
    % Setup hyperref package
    \hypersetup{
      breaklinks=true,  % so long urls are correctly broken across lines
      colorlinks=true,
      urlcolor=blue,
      linkcolor=darkorange,
      citecolor=darkgreen,
      }
    % Slightly bigger margins than the latex defaults
    
    \geometry{verbose,tmargin=1in,bmargin=1in,lmargin=1in,rmargin=1in}
    
    

    \begin{document}
    
    
    \maketitle
    
    

    
    \begin{itemize}
\itemsep1pt\parskip0pt\parsep0pt
\item
  Author: Ren Zhang\\
\item
  email: zhang\_ren@bentley.edu
\end{itemize}

    \subsection{Reference}\label{reference}

\begin{enumerate}
\def\labelenumi{\arabic{enumi}.}
\itemsep1pt\parskip0pt\parsep0pt
\item
  Dataset:

  \begin{itemize}
  \itemsep1pt\parskip0pt\parsep0pt
  \item
    I am using the improved data set downloaded from the Udacity
    website. A
    \href{https://s3.amazonaws.com/uploads.hipchat.com/23756/665149/05bgLZqSsMycnkg/turnstile-weather-variables.pdf}{link}
    to a description of the variables in the data set
  \end{itemize}
\item
  Forum Posts:

  \begin{itemize}
  \itemsep1pt\parskip0pt\parsep0pt
  \item
    \href{https://discussions.udacity.com/t/mann-whitney-u-test-on-improved-dataset-yields-p-nan/4470}{Mann-Whitney
    U Test on improved dataset yields p=NaN?}
  \item
    \href{https://discussions.udacity.com/t/project-mann-whitney-u-test-p-value/18145/6}{Project
    Mann-Whitney U Test p-value}
  \end{itemize}
\item
  Wikipedia Articles:

  \begin{itemize}
  \itemsep1pt\parskip0pt\parsep0pt
  \item
    \href{https://en.wikipedia.org/wiki/Mann\%E2\%80\%93Whitney_U_test}{Mann--Whitney
    U test}
  \end{itemize}
\item
  Book:

  \begin{itemize}
  \itemsep1pt\parskip0pt\parsep0pt
  \item
    \href{http://www-stat.stanford.edu/~tibs/ElemStatLearn/index.html}{Elements
    of Statistical Learning: data mining, inference and prediction} by
    Hastie, Tibshirani and Friedman
  \end{itemize}
\end{enumerate}

    \subsection{Statistical Test}\label{statistical-test}

The variable we are interested in analyzing is \texttt{ENTRIESn\_hourly}
in the dataset. We want to investigate whether the mean rank value of
\texttt{ENTRIESn\_hourly} in a sample of rainny days and the mean rank
value of \texttt{ENTRIESn\_hourly} in a sample of days without rain are
equally likely to be greater than the other.

Judging from the two histograms(see Appendix) of
\texttt{ENTRIESn\_hourly} in these two situations, we find out that the
distributions are not normal. We decide to perform a two-tail
Mann-Whitney U test, because this test does not make any assumptions on
distributions whereas t-test can only be applied on normal
distributions.

\begin{itemize}
\itemsep1pt\parskip0pt\parsep0pt
\item
  The null hypothesis is: the mean rank value of
  \texttt{ENTRIESn\_hourly} in a random sample drawn from the population
  of rainny days and the mean rank value of \texttt{ENTRIESn\_hourly} in
  a random sample drawn from the population of days without rain are
  equally likely to be greater than the other one.\\
\item
  The alternative hypothesis is: the mean rank value of
  \texttt{ENTRIESn\_hourly} in a random sample drawn from the population
  of rainny days and the mean rank value of \texttt{ENTRIESn\_hourly} in
  a random sample drawn from the population of days without rain are not
  equally likely to be greater than the other one.
\end{itemize}

The mean value of \texttt{ENTRIESn\_hourly} on rainny days is 2028.1960
and the mean value of \texttt{ENTRIESn\_hourly} on days without rain is
1845.5394. The U value is 153635120.5 and the associated p value is
5.4827e-06. The p value is very small, even if we use a very small
critical value 0.0001, we can still reject the null hypothesis.

    \subsection{Linear Regression}\label{linear-regression}

We used lasso regression to perform feature selection and model fitting
at the same time.

\begin{quote}
The Lasso is a shrinkage and selection method for linear regression. It
minimizes the usual sum of squared errors, with a bound on the sum of
the absolute values of the coefficients.
\end{quote}

Lasso regression uses a $l_1$ penlty that has the effect of forcing some
of the coefficient estimates to be exactly equal to zero when the tuning
parameter $\lambda$ is sufficiently large. Thus, in the resulting model,
only subset of the features are used.

Since lasso will perform feature selection for us, we can put as many
features as possible into the fitting. The features entered to fit a
lasso regression model are: \texttt{precipi}, \texttt{pressurei},
\texttt{tempi}, \texttt{wspdi}, \texttt{meanprecipi},
\texttt{meanpressurei}, \texttt{meantempi}, \texttt{meanwspdi},
\texttt{hour}, \texttt{day\_week}, \texttt{weekday}, \texttt{conds},
\texttt{rain}, \texttt{fog} and \texttt{UNIT}. Among them, we performed
dummy variable transformations on \texttt{hour}, \texttt{day\_week},
\texttt{weekday}, \texttt{conds}, \texttt{rain}, \texttt{fog} and
\texttt{UNIT}, becasue these variables are categorical rather than
numerical.

We fitted the lasso regression model using a tuning parameter of 0.2 and
the resulting $R^2$ value is 0.5125. The features selected in the model
and the associated coefficients are shown in the Appendix. Unfortuntly,
only the dummy variables remained in the model.

The $R^2$ value 0.5125 means that 51.25\% of variability in the
dependent variable \texttt{ENTRIESn\_hourly} in the sample we have is
explained by the lasso regression model. This $R^2$ value is moderate
and the model is not appropriate for making perdictions on ridership.
Also notice that the $R^2$ is not an estimate for the whole population,
so making perdictions using this model for out of sample data points
might be even more inaccurate.

    \subsection{Visualization}\label{visualization}

The first visualization is the histograms of the dependent variable
\texttt{ENTRIESn\_hourly} on rainny days and on days without rain (see
Appendix for the plots). Judging from the histograms, the distributions
are not normally distributed but rather following a power law
distribution.

The second visualization is a bar plot of the average value of the
dependent variable \texttt{ENTRIESn\_hourly} faceted by the day of week.
We can clearly see that the ridership during weekend is much less than
the ridership during weekdays.

    \subsection{Conclusion}\label{conclusion}

Based on the result of the statistical test, we conclude that more
people ride the NYC subway when it is raining than when it is not
raining. If indeed there is no difference in ridership between rainny
days and days without rain, the chance that we obtain the data we have
in hand is too small that no reasonable people would attribute that to
chance of sampling.

Since the lasso regression model we built did not include the variable
\texttt{rain} as a explanatory variable, so the previous conclusion we
have is based solely on the statistical test. Also, we construced a
random forest regression model and looked at the feature importance
provided by the model (see Appendix), \texttt{rain} is not among the 20
most important variabels.

    \subsection{Reflection}\label{reflection}

One potential shortcomming of this analysis is that, I did not perform
any test about the outliers. Outliers may have great impact on model
fitting.

The second thing is that, $R^2$ is not a good measure when comparing
models with different numbers of features. Adding more explanatory
variables into a model will never causing the $R^2$ value to decrease.

    \subsection{Appendix Python Code for the
project}\label{appendix-python-code-for-the-project}

\subsubsection{Load libraries and set ipython notebook inline
plotting}\label{load-libraries-and-set-ipython-notebook-inline-plotting}

    \begin{Verbatim}[commandchars=\\\{\}]
{\color{incolor}In [{\color{incolor}7}]:} \PY{k+kn}{import} \PY{n+nn}{os}
        \PY{k+kn}{import} \PY{n+nn}{pandas} \PY{k+kn}{as} \PY{n+nn}{pd}
        \PY{k+kn}{import} \PY{n+nn}{numpy} \PY{k+kn}{as} \PY{n+nn}{np}
        \PY{k+kn}{import} \PY{n+nn}{scipy}
        \PY{k+kn}{from} \PY{n+nn}{sklearn} \PY{k+kn}{import} \PY{n}{linear\PYZus{}model}
        \PY{k+kn}{from} \PY{n+nn}{sklearn.metrics} \PY{k+kn}{import} \PY{n}{r2\PYZus{}score}
        \PY{k+kn}{from} \PY{n+nn}{sklearn.ensemble.forest} \PY{k+kn}{import} \PY{n}{RandomForestRegressor}
        \PY{k+kn}{from} \PY{n+nn}{ggplot} \PY{k+kn}{import} \PY{o}{*}
        \PY{k+kn}{import} \PY{n+nn}{matplotlib.pyplot} \PY{k+kn}{as} \PY{n+nn}{plt}
        \PY{o}{\PYZpc{}}\PY{k}{matplotlib} \PY{n}{inline}
\end{Verbatim}


    \subsubsection{Read in data set}


    \begin{Verbatim}[commandchars=\\\{\}]
{\color{incolor}In [{\color{incolor}2}]:} \PY{n}{filename} \PY{o}{=} \PY{l+s}{\PYZdq{}}\PY{l+s}{turnstile\PYZus{}weather\PYZus{}v2.csv}\PY{l+s}{\PYZdq{}}
        \PY{n}{path} \PY{o}{=} \PY{l+s}{\PYZdq{}}\PY{l+s}{d:}\PY{l+s}{\PYZbs{}}\PY{l+s}{GithubRepos}\PY{l+s}{\PYZbs{}}\PY{l+s}{Udacity}\PY{l+s}{\PYZbs{}}\PY{l+s}{P2}\PY{l+s}{\PYZdq{}}
        \PY{n}{df} \PY{o}{=} \PY{n}{pd}\PY{o}{.}\PY{n}{read\PYZus{}csv}\PY{p}{(}\PY{n}{os}\PY{o}{.}\PY{n}{path}\PY{o}{.}\PY{n}{join}\PY{p}{(}\PY{n}{path}\PY{p}{,}\PY{n}{filename}\PY{p}{)}\PY{p}{)}
\end{Verbatim}


    \subsubsection{Statistical Test}


    get \emph{ENTRIESn\_hourly} in two situations

    \begin{Verbatim}[commandchars=\\\{\}]
{\color{incolor}In [{\color{incolor}3}]:} \PY{n}{rain\PYZus{}ENTRIESn\PYZus{}hourly} \PY{o}{=} \PY{n}{df}\PY{p}{[}\PY{n}{df}\PY{p}{[}\PY{l+s}{\PYZdq{}}\PY{l+s}{rain}\PY{l+s}{\PYZdq{}}\PY{p}{]} \PY{o}{==} \PY{l+m+mi}{1}\PY{p}{]}\PY{p}{[}\PY{l+s}{\PYZdq{}}\PY{l+s}{ENTRIESn\PYZus{}hourly}\PY{l+s}{\PYZdq{}}\PY{p}{]}
        \PY{n}{nonrain\PYZus{}ENTRIESn\PYZus{}hourly} \PY{o}{=} \PY{n}{df}\PY{p}{[}\PY{n}{df}\PY{p}{[}\PY{l+s}{\PYZdq{}}\PY{l+s}{rain}\PY{l+s}{\PYZdq{}}\PY{p}{]} \PY{o}{==} \PY{l+m+mi}{0}\PY{p}{]}\PY{p}{[}\PY{l+s}{\PYZdq{}}\PY{l+s}{ENTRIESn\PYZus{}hourly}\PY{l+s}{\PYZdq{}}\PY{p}{]}
\end{Verbatim}

    Calculating the mean values of \emph{ENTRIESn\_hourly} in rainy days and
in days without rain.

    \begin{Verbatim}[commandchars=\\\{\}]
{\color{incolor}In [{\color{incolor}4}]:} \PY{k}{print} \PY{l+s}{\PYZdq{}}\PY{l+s}{rainny days    |  days without rain}\PY{l+s}{\PYZdq{}}
        \PY{k}{print} \PY{n}{np}\PY{o}{.}\PY{n}{mean}\PY{p}{(}\PY{n}{rain\PYZus{}ENTRIESn\PYZus{}hourly}\PY{p}{)}\PY{p}{,}\PY{l+s}{\PYZdq{}}\PY{l+s}{ | }\PY{l+s}{\PYZdq{}}\PY{p}{,} \PY{n}{np}\PY{o}{.}\PY{n}{mean}\PY{p}{(}\PY{n}{nonrain\PYZus{}ENTRIESn\PYZus{}hourly}\PY{p}{)}
\end{Verbatim}

    \begin{Verbatim}[commandchars=\\\{\}]
rainny days    |  days without rain
2028.19603547  |  1845.53943866
    \end{Verbatim}

    There is strange behavior of the \texttt{mannwhitneyu()} function from
the scipy library when used in windows machines against the improved
data set. This is discussed in a couple of posts (see the Reference
part).\\I used the formula for p value as described in the wikipeida
page to calculate it.

    \begin{Verbatim}[commandchars=\\\{\}]
{\color{incolor}In [{\color{incolor}5}]:} \PY{n}{U}\PY{p}{,} \PY{n}{\PYZus{}} \PY{o}{=} \PY{n}{scipy}\PY{o}{.}\PY{n}{stats}\PY{o}{.}\PY{n}{mannwhitneyu}\PY{p}{(}\PY{n}{rain\PYZus{}ENTRIESn\PYZus{}hourly}\PY{p}{,} \PY{n}{nonrain\PYZus{}ENTRIESn\PYZus{}hourly}\PY{p}{)}
        \PY{n}{mu\PYZus{}u} \PY{o}{=} \PY{n+nb}{len}\PY{p}{(}\PY{n}{rain\PYZus{}ENTRIESn\PYZus{}hourly}\PY{p}{)}\PY{o}{*}\PY{n+nb}{len}\PY{p}{(}\PY{n}{nonrain\PYZus{}ENTRIESn\PYZus{}hourly}\PY{p}{)}\PY{o}{/}\PY{l+m+mi}{2}
        \PY{n}{sigma\PYZus{}u} \PY{o}{=} \PY{n}{np}\PY{o}{.}\PY{n}{sqrt}\PY{p}{(}\PY{n+nb}{len}\PY{p}{(}\PY{n}{rain\PYZus{}ENTRIESn\PYZus{}hourly}\PY{p}{)}\PY{o}{*}\PY{n+nb}{len}\PY{p}{(}\PY{n}{nonrain\PYZus{}ENTRIESn\PYZus{}hourly}\PY{p}{)}\PY{o}{*}
                          \PY{p}{(}\PY{n+nb}{len}\PY{p}{(}\PY{n}{rain\PYZus{}ENTRIESn\PYZus{}hourly}\PY{p}{)}\PY{o}{+}\PY{n+nb}{len}\PY{p}{(}\PY{n}{nonrain\PYZus{}ENTRIESn\PYZus{}hourly}\PY{p}{)}\PY{o}{+}\PY{l+m+mi}{1}\PY{p}{)}\PY{o}{/}\PY{l+m+mi}{12}\PY{p}{)}
        \PY{n}{z} \PY{o}{=} \PY{p}{(}\PY{n}{U} \PY{o}{\PYZhy{}} \PY{n}{mu\PYZus{}u}\PY{p}{)}\PY{o}{/}\PY{n}{sigma\PYZus{}u}
        \PY{n}{p} \PY{o}{=} \PY{l+m+mi}{2}\PY{o}{*}\PY{n}{scipy}\PY{o}{.}\PY{n}{stats}\PY{o}{.}\PY{n}{norm}\PY{o}{.}\PY{n}{cdf}\PY{p}{(}\PY{n}{z}\PY{p}{)}
        \PY{k}{print} \PY{l+s}{\PYZdq{}}\PY{l+s}{U value      |  p value}\PY{l+s}{\PYZdq{}}
        \PY{k}{print} \PY{n}{U}\PY{p}{,}\PY{l+s}{\PYZdq{}}\PY{l+s}{ | }\PY{l+s}{\PYZdq{}}\PY{p}{,} \PY{n}{p}
\end{Verbatim}

    \begin{Verbatim}[commandchars=\\\{\}]
U value      |  p value
153635120.5  |  5.48269387142e-06
    \end{Verbatim}

    \subsubsection{Linear regression}\label{linear-regression}

Prepare the features and training labels.

    \begin{Verbatim}[commandchars=\\\{\}]
{\color{incolor}In [{\color{incolor}6}]:} \PY{n}{features\PYZus{}start\PYZus{}with} \PY{o}{=} \PY{p}{[}\PY{l+s}{\PYZdq{}}\PY{l+s}{precipi}\PY{l+s}{\PYZdq{}}\PY{p}{,}\PY{l+s}{\PYZdq{}}\PY{l+s}{pressurei}\PY{l+s}{\PYZdq{}}\PY{p}{,}\PY{l+s}{\PYZdq{}}\PY{l+s}{tempi}\PY{l+s}{\PYZdq{}}\PY{p}{,}\PY{l+s}{\PYZdq{}}\PY{l+s}{wspdi}\PY{l+s}{\PYZdq{}}\PY{p}{,}
                                \PY{l+s}{\PYZdq{}}\PY{l+s}{meanprecipi}\PY{l+s}{\PYZdq{}}\PY{p}{,}\PY{l+s}{\PYZdq{}}\PY{l+s}{meanpressurei}\PY{l+s}{\PYZdq{}}\PY{p}{,}\PY{l+s}{\PYZdq{}}\PY{l+s}{meantempi}\PY{l+s}{\PYZdq{}}\PY{p}{,}\PY{l+s}{\PYZdq{}}\PY{l+s}{meanwspdi}\PY{l+s}{\PYZdq{}}\PY{p}{,}\PY{p}{]}
        
        \PY{n}{categorical\PYZus{}features\PYZus{}to\PYZus{}dummies} \PY{o}{=} \PY{p}{[}\PY{l+s}{\PYZdq{}}\PY{l+s}{hour}\PY{l+s}{\PYZdq{}}\PY{p}{,}\PY{l+s}{\PYZdq{}}\PY{l+s}{day\PYZus{}week}\PY{l+s}{\PYZdq{}}\PY{p}{,}\PY{l+s}{\PYZdq{}}\PY{l+s}{weekday}\PY{l+s}{\PYZdq{}}\PY{p}{,}\PY{l+s}{\PYZdq{}}\PY{l+s}{conds}\PY{l+s}{\PYZdq{}}\PY{p}{,}\PY{l+s}{\PYZdq{}}\PY{l+s}{rain}\PY{l+s}{\PYZdq{}}\PY{p}{,}\PY{l+s}{\PYZdq{}}\PY{l+s}{fog}\PY{l+s}{\PYZdq{}}\PY{p}{,}\PY{l+s}{\PYZdq{}}\PY{l+s}{UNIT}\PY{l+s}{\PYZdq{}}\PY{p}{]}
        
        \PY{n}{X} \PY{o}{=} \PY{n}{df}\PY{p}{[}\PY{n}{features\PYZus{}start\PYZus{}with}\PY{p}{]}
        
        \PY{k}{for} \PY{n}{feature} \PY{o+ow}{in} \PY{n}{categorical\PYZus{}features\PYZus{}to\PYZus{}dummies}\PY{p}{:}
            \PY{n}{dummy} \PY{o}{=} \PY{n}{pd}\PY{o}{.}\PY{n}{get\PYZus{}dummies}\PY{p}{(}\PY{n}{df}\PY{p}{[}\PY{n}{feature}\PY{p}{]}\PY{p}{,} \PY{n}{prefix} \PY{o}{=} \PY{n}{feature}\PY{p}{)}
            \PY{n}{X} \PY{o}{=} \PY{n}{X}\PY{o}{.}\PY{n}{join}\PY{p}{(}\PY{n}{dummy}\PY{p}{)}
            
        \PY{n}{y} \PY{o}{=} \PY{n}{df}\PY{p}{[}\PY{l+s}{\PYZsq{}}\PY{l+s}{ENTRIESn\PYZus{}hourly}\PY{l+s}{\PYZsq{}}\PY{p}{]}
\end{Verbatim}

    Fit lasso regression model

    \begin{Verbatim}[commandchars=\\\{\}]
{\color{incolor}In [{\color{incolor}7}]:} \PY{n}{clf} \PY{o}{=} \PY{n}{linear\PYZus{}model}\PY{o}{.}\PY{n}{Lasso}\PY{p}{(}\PY{n}{alpha}\PY{o}{=}\PY{l+m+mf}{0.2}\PY{p}{,} \PY{n}{fit\PYZus{}intercept} \PY{o}{=} \PY{n+nb+bp}{True}\PY{p}{,} \PY{n}{normalize} \PY{o}{=} \PY{n+nb+bp}{True}\PY{p}{)}
        \PY{n}{clf}\PY{o}{.}\PY{n}{fit}\PY{p}{(}\PY{n}{X}\PY{p}{,}\PY{n}{y}\PY{p}{)}
\end{Verbatim}

            \begin{Verbatim}[commandchars=\\\{\}]
{\color{outcolor}Out[{\color{outcolor}7}]:} Lasso(alpha=0.2, copy\_X=True, fit\_intercept=True, max\_iter=1000,
           normalize=True, positive=False, precompute=False, random\_state=None,
           selection='cyclic', tol=0.0001, warm\_start=False)
\end{Verbatim}
        
    Make prediction on the sample data and calculate the R Squared value

    \begin{Verbatim}[commandchars=\\\{\}]
{\color{incolor}In [{\color{incolor}8}]:} \PY{n}{y\PYZus{}pred} \PY{o}{=} \PY{n}{clf}\PY{o}{.}\PY{n}{predict}\PY{p}{(}\PY{n}{X}\PY{p}{)}
        \PY{n}{r2\PYZus{}score}\PY{p}{(}\PY{n}{y}\PY{p}{,}\PY{n}{clf}\PY{o}{.}\PY{n}{predict}\PY{p}{(}\PY{n}{X}\PY{p}{)}\PY{p}{)}
\end{Verbatim}

            \begin{Verbatim}[commandchars=\\\{\}]
{\color{outcolor}Out[{\color{outcolor}8}]:} 0.51248812846310254
\end{Verbatim}
        
    See the variables used in the model as well as the coefficients

    \begin{Verbatim}[commandchars=\\\{\}]
{\color{incolor}In [{\color{incolor}9}]:} \PY{n}{names} \PY{o}{=} \PY{n}{X}\PY{o}{.}\PY{n}{columns}\PY{p}{[}\PY{n}{clf}\PY{o}{.}\PY{n}{coef\PYZus{}} \PY{o}{\PYZgt{}} \PY{l+m+mf}{0.01}\PY{p}{]}
        \PY{n}{coefficients} \PY{o}{=} \PY{n}{clf}\PY{o}{.}\PY{n}{coef\PYZus{}}\PY{p}{[}\PY{n}{clf}\PY{o}{.}\PY{n}{coef\PYZus{}} \PY{o}{\PYZgt{}} \PY{l+m+mf}{0.01}\PY{p}{]}
        \PY{n}{row\PYZus{}name}\PY{p}{,} \PY{n}{coef\PYZus{}value} \PY{o}{=} \PY{l+s}{\PYZdq{}}\PY{l+s}{\PYZdq{}}\PY{p}{,}\PY{l+s}{\PYZdq{}}\PY{l+s}{\PYZdq{}}
        \PY{k}{for} \PY{n}{i} \PY{o+ow}{in} \PY{n+nb}{range}\PY{p}{(}\PY{n+nb}{len}\PY{p}{(}\PY{n}{names}\PY{p}{)}\PY{p}{)}\PY{p}{:}
            \PY{n}{row\PYZus{}name} \PY{o}{+}\PY{o}{=} \PY{n}{names}\PY{p}{[}\PY{n}{i}\PY{p}{]}\PY{o}{.}\PY{n}{rjust}\PY{p}{(}\PY{l+m+mi}{12}\PY{p}{)}
            \PY{n}{coef\PYZus{}value} \PY{o}{+}\PY{o}{=} \PY{n+nb}{str}\PY{p}{(}\PY{n+nb}{round}\PY{p}{(}\PY{n}{coefficients}\PY{p}{[}\PY{n}{i}\PY{p}{]}\PY{p}{,}\PY{l+m+mi}{5}\PY{p}{)}\PY{p}{)}\PY{o}{.}\PY{n}{rjust}\PY{p}{(}\PY{l+m+mi}{12}\PY{p}{)}
            \PY{k}{if} \PY{n+nb}{len}\PY{p}{(}\PY{n}{row\PYZus{}name}\PY{p}{)} \PY{o}{\PYZgt{}} \PY{l+m+mi}{80}\PY{p}{:}
                \PY{k}{print} \PY{n}{row\PYZus{}name}
                \PY{k}{print} \PY{n}{coef\PYZus{}value}
                \PY{k}{print} 
                \PY{n}{row\PYZus{}name}\PY{p}{,} \PY{n}{coef\PYZus{}value} \PY{o}{=} \PY{l+s}{\PYZdq{}}\PY{l+s}{\PYZdq{}}\PY{p}{,}\PY{l+s}{\PYZdq{}}\PY{l+s}{\PYZdq{}}
\end{Verbatim}

    \begin{Verbatim}[commandchars=\\\{\}]
hour\_12     hour\_20  day\_week\_3 conds\_Clear   UNIT\_R011   UNIT\_R012   UNIT\_R013
   705.82337   912.87137     0.75025    42.73925  5401.16449  6753.92547    652.3663

   UNIT\_R017   UNIT\_R018   UNIT\_R019   UNIT\_R020   UNIT\_R021   UNIT\_R022   UNIT\_R023
  2267.36089  5913.86907   1398.4282  4443.45774  2752.76777  7587.85015  4223.02754

   UNIT\_R024   UNIT\_R025   UNIT\_R027   UNIT\_R029   UNIT\_R030   UNIT\_R031   UNIT\_R032
  1358.29953  3495.32068  1037.46312   5299.4362  1169.62972  2421.39851  2515.19138

   UNIT\_R033   UNIT\_R035   UNIT\_R041   UNIT\_R043   UNIT\_R044   UNIT\_R046   UNIT\_R049
  6304.40921   866.38262  1169.44144    956.5758  2747.98437  6414.43595   842.24775

   UNIT\_R050   UNIT\_R051   UNIT\_R053   UNIT\_R055   UNIT\_R057   UNIT\_R062   UNIT\_R080
  2093.80474  3203.98963   1308.7658  6473.91472  2952.37678   806.23692  1681.54336

   UNIT\_R081   UNIT\_R083   UNIT\_R084   UNIT\_R085   UNIT\_R086   UNIT\_R092   UNIT\_R093
  1628.34557  1195.95191  8100.22607   678.76364   669.54322   122.42588   155.24393

   UNIT\_R095   UNIT\_R096   UNIT\_R097   UNIT\_R099   UNIT\_R101   UNIT\_R102   UNIT\_R105
   312.92751   499.53521  1122.90662   461.12943   895.52192   1784.3606   1434.0595

   UNIT\_R108   UNIT\_R111   UNIT\_R116   UNIT\_R119   UNIT\_R122   UNIT\_R127   UNIT\_R137
  3323.22618  1320.34981  1299.49494     4.28494   727.18736  2902.11328   619.82327

   UNIT\_R139   UNIT\_R163   UNIT\_R172   UNIT\_R179   UNIT\_R188   UNIT\_R194   UNIT\_R198
   635.79067  1437.84983     3.88743   4884.6616   436.89373    91.82522   219.21858

   UNIT\_R202   UNIT\_R207   UNIT\_R208   UNIT\_R211   UNIT\_R218   UNIT\_R223   UNIT\_R235
   401.42173   112.64656   684.90573   522.81744   165.34377   303.48108   688.61001
    \end{Verbatim}

    \subsubsection{Visualization}\label{visualization}

    create histogram for \emph{ENTRIESn\_hourly} on rainy days and on days
without rain.

    \begin{Verbatim}[commandchars=\\\{\}]
{\color{incolor}In [{\color{incolor}11}]:} \PY{n}{ggplot}\PY{p}{(}\PY{n}{aes}\PY{p}{(}\PY{n}{x}\PY{o}{=}\PY{l+s}{\PYZdq{}}\PY{l+s}{ENTRIESn\PYZus{}hourly}\PY{l+s}{\PYZdq{}}\PY{p}{,} \PY{n}{color} \PY{o}{=} \PY{l+s}{\PYZdq{}}\PY{l+s}{rain}\PY{l+s}{\PYZdq{}}\PY{p}{)} \PY{p}{,} \PY{n}{data} \PY{o}{=} \PY{n}{df}\PY{p}{)} \PY{o}{+} \PYZbs{}
             \PY{n}{geom\PYZus{}histogram}\PY{p}{(}\PY{n}{binwidth} \PY{o}{=} \PY{l+m+mi}{300}\PY{p}{,} \PY{n}{alpha} \PY{o}{=} \PY{l+m+mf}{0.9}\PY{p}{)} \PY{o}{+} \PYZbs{}
             \PY{n}{ylab}\PY{p}{(}\PY{l+s}{\PYZdq{}}\PY{l+s}{Count}\PY{l+s}{\PYZdq{}}\PY{p}{)} \PY{o}{+} \PYZbs{}
             \PY{n}{ggtitle}\PY{p}{(}\PY{l+s}{\PYZdq{}}\PY{l+s}{Ridership on Rainny Days}\PY{l+s}{\PYZdq{}}\PY{p}{)} \PY{o}{+} \PYZbs{}
             \PY{n}{xlim}\PY{p}{(}\PY{l+m+mi}{0}\PY{p}{,}\PY{l+m+mi}{15000}\PY{p}{)}
\end{Verbatim}

    \begin{center}
    \adjustimage{max size={0.9\linewidth}{0.9\paperheight}}{Analyzing the NYC Subway Dataset_files/Analyzing the NYC Subway Dataset_28_0.png}
    \end{center}
    { \hspace*{\fill} \\}
    
            \begin{Verbatim}[commandchars=\\\{\}]
{\color{outcolor}Out[{\color{outcolor}11}]:} <ggplot: (21003019)>
\end{Verbatim}
        
    Both distributions are more likly following the power law distribution
rather than normal distribution.

    \begin{Verbatim}[commandchars=\\\{\}]
{\color{incolor}In [{\color{incolor}15}]:} \PY{n}{grouped} \PY{o}{=} \PY{n}{df}\PY{o}{.}\PY{n}{groupby}\PY{p}{(}\PY{l+s}{\PYZdq{}}\PY{l+s}{day\PYZus{}week}\PY{l+s}{\PYZdq{}}\PY{p}{)}
         \PY{n}{avg\PYZus{}entri\PYZus{}by\PYZus{}weekday} \PY{o}{=} \PY{n}{grouped}\PY{p}{[}\PY{l+s}{\PYZdq{}}\PY{l+s}{ENTRIESn\PYZus{}hourly}\PY{l+s}{\PYZdq{}}\PY{p}{]}\PY{o}{.}\PY{n}{mean}\PY{p}{(}\PY{p}{)}
         \PY{n}{plt}\PY{o}{.}\PY{n}{figure}\PY{p}{(}\PY{p}{)}
         \PY{n}{avg\PYZus{}entri\PYZus{}by\PYZus{}weekday}\PY{o}{.}\PY{n}{plot}\PY{p}{(}\PY{n}{kind} \PY{o}{=} \PY{l+s}{\PYZdq{}}\PY{l+s}{bar}\PY{l+s}{\PYZdq{}}\PY{p}{)}
         \PY{n}{title}\PY{p}{(}\PY{l+s}{\PYZdq{}}\PY{l+s}{Average Ridership per day}\PY{l+s}{\PYZdq{}}\PY{p}{)}
         \PY{n}{labels} \PY{o}{=} \PY{p}{[}\PY{l+s}{\PYZdq{}}\PY{l+s}{Monday}\PY{l+s}{\PYZdq{}}\PY{p}{,}\PY{l+s}{\PYZdq{}}\PY{l+s}{Tuesday}\PY{l+s}{\PYZdq{}}\PY{p}{,}\PY{l+s}{\PYZdq{}}\PY{l+s}{Wednesday}\PY{l+s}{\PYZdq{}}\PY{p}{,}\PY{l+s}{\PYZdq{}}\PY{l+s}{Thursday}\PY{l+s}{\PYZdq{}}\PY{p}{,}\PY{l+s}{\PYZdq{}}\PY{l+s}{Friday}\PY{l+s}{\PYZdq{}}\PY{p}{,}\PY{l+s}{\PYZdq{}}\PY{l+s}{Saturday}\PY{l+s}{\PYZdq{}}\PY{p}{,}\PY{l+s}{\PYZdq{}}\PY{l+s}{Sunday}\PY{l+s}{\PYZdq{}}\PY{p}{]}
         \PY{n}{plt}\PY{o}{.}\PY{n}{xticks}\PY{p}{(}\PY{n+nb}{range}\PY{p}{(}\PY{l+m+mi}{7}\PY{p}{)}\PY{p}{,} \PY{n}{labels}\PY{p}{,} \PY{n}{rotation}\PY{o}{=}\PY{l+s}{\PYZsq{}}\PY{l+s}{vertical}\PY{l+s}{\PYZsq{}}\PY{p}{)}
         \PY{n}{plt}\PY{o}{.}\PY{n}{show}\PY{p}{(}\PY{p}{)}
\end{Verbatim}

    \begin{center}
    \adjustimage{max size={0.9\linewidth}{0.9\paperheight}}{Analyzing the NYC Subway Dataset_files/Analyzing the NYC Subway Dataset_30_0.png}
    \end{center}
    { \hspace*{\fill} \\}
    
    The two bars on the right hand side are the average value of
\emph{ENTRIESn\_hourly} on saturday and sunday. The bars are relatively
shorter than the bars on weekdays.

    \begin{Verbatim}[commandchars=\\\{\}]
{\color{incolor}In [{\color{incolor}13}]:} \PY{n}{RF\PYZus{}model} \PY{o}{=} \PY{n}{RandomForestRegressor}\PY{p}{(}\PY{n}{random\PYZus{}state}\PY{o}{=}\PY{l+m+mi}{0}\PY{p}{,} \PY{n}{n\PYZus{}estimators}\PY{o}{=}\PY{l+m+mi}{10}\PY{p}{)}
         \PY{n}{RF\PYZus{}model}\PY{o}{.}\PY{n}{fit}\PY{p}{(}\PY{n}{X}\PY{p}{,}\PY{n}{y}\PY{p}{)}
\end{Verbatim}

            \begin{Verbatim}[commandchars=\\\{\}]
{\color{outcolor}Out[{\color{outcolor}13}]:} RandomForestRegressor(bootstrap=True, criterion='mse', max\_depth=None,
                    max\_features='auto', max\_leaf\_nodes=None, min\_samples\_leaf=1,
                    min\_samples\_split=2, min\_weight\_fraction\_leaf=0.0,
                    n\_estimators=10, n\_jobs=1, oob\_score=False, random\_state=0,
                    verbose=0, warm\_start=False)
\end{Verbatim}
        
    The most important 20 features based on random forest model.

    \begin{Verbatim}[commandchars=\\\{\}]
{\color{incolor}In [{\color{incolor}15}]:} \PY{n}{names} \PY{o}{=} \PY{n}{X}\PY{o}{.}\PY{n}{columns}\PY{p}{[}\PY{n}{RF\PYZus{}model}\PY{o}{.}\PY{n}{feature\PYZus{}importances\PYZus{}} \PY{o}{\PYZgt{}}\PY{o}{=} \PY{n}{np}\PY{o}{.}\PY{n}{sort}\PY{p}{(}\PY{n}{RF\PYZus{}model}\PY{o}{.}\PY{n}{feature\PYZus{}importances\PYZus{}}\PY{p}{)}\PY{p}{[}\PY{p}{:}\PY{p}{:}\PY{o}{\PYZhy{}}\PY{l+m+mi}{1}\PY{p}{]}\PY{p}{[}\PY{l+m+mi}{20}\PY{p}{]}\PY{p}{]}
         \PY{k}{print} \PY{n}{names}
\end{Verbatim}

    \begin{Verbatim}[commandchars=\\\{\}]
Index([u'meantempi', u'hour\_0', u'hour\_4', u'hour\_8', u'hour\_12', u'hour\_16',
       u'hour\_20', u'weekday\_0', u'weekday\_1', u'UNIT\_R011', u'UNIT\_R012',
       u'UNIT\_R018', u'UNIT\_R020', u'UNIT\_R022', u'UNIT\_R023', u'UNIT\_R029',
       u'UNIT\_R033', u'UNIT\_R046', u'UNIT\_R055', u'UNIT\_R084', u'UNIT\_R179'],
      dtype='object')
    \end{Verbatim}


    % Add a bibliography block to the postdoc
    
    
    
    \end{document}
